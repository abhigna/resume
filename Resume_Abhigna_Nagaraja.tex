% resume.tex
% vim:set ft=tex spell:

\documentclass[11pt,letterpaper,hidelinks]{article}
\usepackage[letterpaper,margin=0.2in]{geometry}
\usepackage[utf8]{inputenc}
\usepackage{mdwlist}
\usepackage[T1]{fontenc}
\usepackage{textcomp}
\usepackage{tgpagella}
\usepackage{savetrees}
\usepackage[colorlinks = true,
            linkcolor = blue,
            urlcolor  = blue,
            citecolor = blue,
            anchorcolor = blue]{hyperref}
\usepackage{fancyhdr}
\usepackage{xcolor}
\usepackage{graphicx}
\usepackage{marvosym}

\newcommand{\changeurlcolor}[1]{\hypersetup{urlcolor=#1}}      

\pagestyle{fancy}
\pagestyle{empty}
\setlength{\tabcolsep}{0em}

% indentsection style, used for sections that aren't already in lists
% that need indentation to the level of all text in the document
\newenvironment{indentsection}[1]%
{\begin{list}{}%
	{\setlength{\leftmargin}{#1}}%
	\item[]%
}
{\end{list}}

% opposite of above; bump a section back toward the left margin
\newenvironment{unindentsection}[1]%
{\begin{list}{}%
	{\setlength{\leftmargin}{-0.5#1}}%
	\item[]%
}
{\end{list}}

% format two pieces of text, one left aligned and one right aligned
\newcommand{\headerrow}[2]
{\begin{tabular*}{\linewidth}{l@{\extracolsep{\fill}}r}
	#1 &
	#2 \\
\end{tabular*}}

% make "C++" look pretty when used in text by touching up the plus signs
\newcommand{\CPP}
{C\nolinebreak[4]\hspace{-.05em}\raisebox{.22ex}{\footnotesize\bf ++}}


%%%%%%%%%%%%%%%%%%%%%%%%%%%%%%%%%%%%%%%%%%%%%%%%%%%
%%%%%%%%% The actual content starts here   %%%%%%%%%%%%%%%%%%%%%%%%%%
%%%%%%%%%%%%%%%%%%%%%%%%%%%%%%%%%%%%%%%%%%%%%%%%%%%
\begin{document}

\begin{center}
{\LARGE \textbf{Abhigna Nagaraja}}

\vspace{0pt}
\small \raisebox{-0.25ex}\Email\ \href{mailto:abhigna.n4@gmail.com}{abhigna.n4@gmail.com}
\ \textbullet\ \small \raisebox{-0.25ex}{\includegraphics[width=2.5ex]{Octocat.jpeg}} \href{https://www.linkedin.com/in/abhigna-nagaraja}{linkedin.com/in/abhigna-nagaraja}
\ \textbullet\ \small \raisebox{-0.25ex}{\includegraphics[width=2.5ex]{LinkedIn-InBug-2C.jpeg}}\href{https://github.com/abhigna}{github.com/abhigna} 
\ \textbullet\ \small \raisebox{-0.25ex}{\includegraphics[width=2.5ex]{website-favicon.png}} \href{https://abhigna.dev}{abhigna.dev} \\
\small \href{tel:+14013262244}{(401)326-2244}
\ \textbullet\ \small Seattle, WA, USA

\vspace{-6pt}
\end{center}

%%%%%%%%%%%%%%%%%%%%%%%%%%%%%%%%%%%%%%%%%%%%%%%%%%%
%%%%%%%%%%%---OBJECTIVE---%%%%%%%%%%%%%%%%%%%%%%%%%%%%%%%
%%%%%%%%%%%%%%%%%%%%%%%%%%%%%%%%%%%%%%%%%%%%%%%%%%%

%\subsection*{\uppercase{Objective}}
%\vspace{-1pt}
%\begin{itemize}
%\item[] Seeking a software engineering \emph{internship} with the data %science or data engineering team in a company that will allow me to %use my skills in machine learning, big data systems, and %recommendation engines.
%\end{itemize}
%\vspace{2pt}

%%%%%%%%%%%%%%%%%%%%%%%%%%%%%%%%%%%%%%%%%%%%%%%%%%%
%%%%%%%%%%%---EXPERIENCE SECTION---%%%%%%%%%%%%%%%%%%%%%%%%%
%%%%%%%%%%%%%%%%%%%%%%%%%%%%%%%%%%%%%%%%%%%%%%%%%%%

\hrule
\subsection*{\uppercase{Relevant Experience}}
\vspace{-3pt}
\begin{itemize}
	\parskip=0.1em
        \item
	\headerrow
		{\textbf{Meta Platforms, Inc}}
		{\emph{Seattle, US}}
    \\
    \headerrow
		{\emph{Staff Software Engineer , Meta Platforms Inc}}
		{\emph{Oct 2021 -- Present}}
	 	\begin{itemize*}
\item Led the successful migration of 10 year old distributed system (used by 2400+ teams) processing > 24 trillion operations daily in a 2 year+ migration effort
\item Reduced operational events by 60\%, accelerated development of newer capacities, and achieved substantial server savings (95+ million dollars) from the migration
\item Spearheaded the development of a capacity pooling concept that helped 300+ teams optimize their server usage generating \$6 million in annual capacity savings and improving system responsiveness by 10–20\%
\item Designed and implemented a traffic shaping mechanism, resulting in substantial costs savings and resilient operation at traffic peaks
\item Speadheaded a cross-org efficiency initiatve to achieve close to 90+million\$ in capacity savings in 2024,2025. 
\item Colloborated across org with stakeholders on roadmap planning for multiple years
\item Built a generic backpressure mechanism that helped improve reliability of meta products like recommendation systems
\item Positively impacted the reliability of features in Meta products such as WhatsApp, Instagram, Facebook, and Thread.
\item Co-authored the research paper "XFaaS: Hyperscale and Low Cost Serverless Functions," shedding light on the intricacies of serverless computing within hyperscale environments and contributing valuable insights to the field of cloud computing.
		\end{itemize*}
    \
        \item
	\headerrow
		{\textbf{Amazon Web Services}}
		{\emph{Seattle, US}}
    \\
    \headerrow
		{\emph{Sr Software Engineer , Amazon API Gateway}}
		{\emph{April 2019 -- September 2021}}
	 	\begin{itemize*}
	    \item Lead engineering for one of the highest traffic volume AWS service - Amazon API Gateway - a highly configurable reverse proxy with tight integration with many AWS services.
	 	\item Guide API Gateway's architectural vision on scalability, availability and operational posture. Sponsored a cross team re-architectural effort to build a scalable shuffle sharded data plane.
	 	\item Represent API Gateway's engineering in broader AWS-wide initiatives like regular availability risk review.
	 	\item Lead API Gateway's technical strategy in new initiates to support gRPC protocol, private connectivity to APIs, improve rate limiting customer experience for API Gateway's customer (Limitless).
	 	\item Led effort to deliver complex customer impacting features for API Gateway like Mutual TLS (\changeurlcolor{blue}\href{https://aws.amazon.com/blogs/compute/introducing-mutual-tls-authentication-for-amazon-api-gateway/}{official blog post}). Built consensus on approach with AWS-wide stakeholders, designed and led the core development team. Efforts led to the successful delivery of the critical requested feature for enterprise customers.\
	 	\item Drive API Gateway's architectural vision to separate frontend layer into a separate layer and led the effort to build the initial version. Built consensus on approach with cross-team stakeholders, developed and delivered a Rust based frontend layer for API Gateway that handles an initial set of usecases with a vision to solve future product usecases.
	 	\item Led API Gateway's engineering effort to evaluate the next gen data plane technology. This resulted in the new platform that powers the HTTP API platform (\changeurlcolor{blue}\href{https://aws.amazon.com/blogs/compute/announcing-http-apis-for-amazon-api-gateway/}{official blog post})
	 	\item Designed and built a custom networking protocol using TUN to eliminate a vertical scaling cliff in API Gateway's internal mechanism to support VPC integration. The mechanism eliminated the scaling issue entirely which resulted in a much simpler architecture.
		\end{itemize*}
    \
    \item
	\headerrow
		{\textbf{Amazon Web Services}}
		{\emph{Seattle, US}}
    \\
    \headerrow
		{\emph{Software Development Engineer II, Amazon API Gateway}}
		{\emph{Feb 2016 -- March 2019}}
	 	\begin{itemize*}
	 	\item Led the effort to build a new stateful data plane to support WebSocket protocol in API Gateway (\changeurlcolor{blue}\href{https://aws.amazon.com/blogs/compute/announcing-websocket-apis-in-amazon-api-gateway/}{official blog post}). Designed and developed the core pieces of data plane while mentoring junior engineers. WebSocket data plane handles some of API Gateway's largest customers.
        \item Led the effort to deliver Private API for API Gateway (\changeurlcolor{blue}\href{https://aws.amazon.com/blogs/compute/introducing-amazon-api-gateway-private-endpoints/}{official blog post}). Scoped requirements with Product, designed and developed a new data plane to handle the Private API traffic.
        \item Helped deliver VPC integration for API Gateway (\changeurlcolor{blue}\href{https://aws.amazon.com/blogs/compute/introducing-amazon-api-gateway-private-endpoints/}{official blog post}). Developed the low level design and delivered feature as part of a team.
        \item Led the effort to deliver customization of error responses for API Gateway (\changeurlcolor{blue}\href{https://aws.amazon.com/about-aws/whats-new/2017/06/amazon-api-gateway-enables-customization-of-error-responses/}{official blog post}). Scoped the requirements, designed and developed the Gateway Responses feature that helped with customizing errors from API Gateway.
		\end{itemize*}
    \item
	\headerrow
		{\textbf{Amazon}}
		{\emph{Bangalore, India}}
    \\
    \headerrow
		{\emph{Software Development Engineer II, Product Aggregator}}
		{\emph{Sept 2015 -- Jan 2016}}
	 	\begin{itemize*}
	 	\item Led effort for utilization improvements on a high throughput web service that's on the critical path for all product page loads. Re-architected the caching mechanism which resulted is significant capex savings for the service.
		\end{itemize*}
		
    \item
	\headerrow
		{\textbf{Amazon}}
		{\emph{Bangalore, India}}
    \\
    \headerrow
		{\emph{Software Developer Engineer I, Product Aggregator}}
		{\emph{April 2013 -- Aug 2015}}
	 	\begin{itemize*}
	 	\item Developed an internal web portal to help analyse traffic patterns from hundreds of internal callers. Its used by the team during operational events to look at traffic patterns.
        \\
    	{\textbf{Tools and Technologies:} { Java, Perl, Python}}
		\end{itemize*}
	 
	 	\item
	\headerrow
		{\textbf{Akamai}}
		{\emph{Bengalore, India}}
    \\
    \headerrow
		{\emph{Associate Solutions Architect, APAC}}
		{\emph{July 2012 -- April 2013}}
	  \begin{itemize*}
	 	\item Delivered integration solutions for Akamai customers by evaluating the web architecture, security requirements, DNS configuration that satisfies the customer's business and technical requirements.
		\end{itemize*}
\end{itemize}
\vspace{3pt}


%%%%%%%%%%%%%%%%%%%%%%%%%%%%%%%%%%%%%%%%%%%%%%%%%%%
%%%%%%%%%%%---EDUCATION---%%%%%%%%%%%%%%%%%%%%%%%%%%%%%%%
%%%%%%%%%%%%%%%%%%%%%%%%%%%%%%%%%%%%%%%%%%%%%%%%%%%

\hrule
%\vspace{1pt}
\subsection*{\uppercase{Education}}
\vspace{-3pt}
\begin{itemize}
	\parskip=0.1em
    \item 
	\headerrow
		{\href{http://www.msrit.edu/}{\textbf{Ramaiah Institute of Technology}}}
		{\emph{Bengaluru, India}}
	\\
	\headerrow
		{\emph{B.E., Information Science and Engineering} \hspace{20 mm} \textbf{GPA - 9.28 / 10}} 
		{\emph{2008 - 2012}}
\end{itemize}
\vspace{3pt}


\end{document}
